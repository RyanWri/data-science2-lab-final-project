
\documentclass[a4paper,landscape]{article}
\usepackage{amssymb,amsmath,amsthm,amsfonts}
\usepackage{multicol}
\usepackage{calc}
\usepackage{ifthen}
\usepackage[left=1cm, right=1cm, top=1cm, bottom=1cm]{geometry}
\usepackage{etoolbox} % For patching commands
\usepackage{longtable} % for tables support
\usepackage{microtype} % For fine control over letter spacing


% enable shading (for code blocks)


\usepackage{fontspec} % Required for polyglossia
\setmainfont{DejaVu Sans} % DejaVu Sans supports both Hebrew and English

% Set the main languages
\usepackage{polyglossia} % Better multilingual support
\setmainlanguage{english}
\setotherlanguage{hebrew}

\newfontfamily\hebrewfont{DejaVu Sans}
\newfontfamily\englishfont{DejaVu Sans}

\ifthenelse{\lengthtest { \paperwidth = 11in}}
    { \geometry{top=.5in,left=.5in,right=.5in,bottom=.5in} }
	{\ifthenelse{ \lengthtest{ \paperwidth = 297mm}}
		{\geometry{top=1cm,left=1cm,right=1cm,bottom=1cm} }
		{\geometry{top=1cm,left=1cm,right=1cm,bottom=1cm} }
	}


% Global settings for table text
\AtBeginEnvironment{tabular}{%
  \addfontfeatures{LetterSpace=-1.0} % Reduce letter spacing
  \small % Make text smaller
}

\pagestyle{empty}
\makeatletter
\renewcommand{\section}{\@startsection{section}{1}{0mm}%
                                {-1ex plus -.5ex minus -.2ex}%
                                {0.5ex plus .2ex}%x
                                {\normalfont\large\bfseries}}
\renewcommand{\subsection}{\@startsection{subsection}{2}{0mm}%
                                {-1explus -.5ex minus -.2ex}%
                                {0.5ex plus .2ex}%
                                {\normalfont\normalsize\bfseries}}
\renewcommand{\subsubsection}{\@startsection{subsubsection}{3}{0mm}%
                                {-1ex plus -.5ex minus -.2ex}%
                                {1ex plus .2ex}%
                                {\normalfont\small\bfseries}}
                          
                                
\makeatother
\setcounter{secnumdepth}{0}
\setlength{\parindent}{0pt}
\setlength{\parskip}{0pt plus 0.5ex}

% Define \tightlist 
\providecommand{\tightlist}{%
  \setlength{\itemsep}{0pt}\setlength{\parskip}{0pt}}

\title{Statistics cheat sheet}

\begin{document}

\raggedright
% define font size
\footnotesize
%\small



\setlength{\premulticols}{1pt}
\setlength{\postmulticols}{1pt}
\setlength{\multicolsep}{1pt}
\setlength{\columnsep}{2pt}
\setlength{\columnsep}{0.5cm} % Adjust column separation

\section{Task 9 - EDA}\label{task-9---eda}

We used translation for our columns, you can see them in translation
folder.

We built a generic EDA (exploratory data analysis) for the common good.

\subsection{Team 2 - Son and Gal}\label{team-2---son-and-gal}

\subsubsection{Tasks:}\label{tasks}

\begin{longtable}[]{@{}
  >{\raggedright\arraybackslash}p{(\columnwidth - 6\tabcolsep) * \real{0.0592}}
  >{\raggedright\arraybackslash}p{(\columnwidth - 6\tabcolsep) * \real{0.4474}}
  >{\raggedright\arraybackslash}p{(\columnwidth - 6\tabcolsep) * \real{0.2763}}
  >{\raggedright\arraybackslash}p{(\columnwidth - 6\tabcolsep) * \real{0.2171}}@{}}
\toprule\noalign{}
\begin{minipage}[b]{\linewidth}\raggedright
Task ID
\end{minipage} & \begin{minipage}[b]{\linewidth}\raggedright
Description
\end{minipage} & \begin{minipage}[b]{\linewidth}\raggedright
Status
\end{minipage} & \begin{minipage}[b]{\linewidth}\raggedright
Progress
\end{minipage} \\
\midrule\noalign{}
\endhead
\bottomrule\noalign{}
\endlastfoot
2 & Create a full Word file and Jupyter notebook & Written when everyone
finishes their tasks. & Not Started \\
9 & EDA of hospitalization1 & \textbf{Finished} & 100\% \\
20 & Connection between Doctor Type and rehospitalization & Created
Classification model, considered adjustments to the data. & 80\% \\
26 & Connection between age and gender to rehospitalization from 16-19 &
Pending for tasks 16-18 & Not Started \\
38 & Dimension Reduction for hospitalization2 & Pending for task 20's
pull request & 60\% \\
46 & Editing conclusions chapter & Pending & Not Started \\
Unique & PowerPoint Presentation & Pending & Not Started \\
\end{longtable}

\subsubsection{Conclusions:}\label{conclusions}

\begin{verbatim}
**Task 9: **
Translations to hebrew, multiple diagnoses - top 10 dignoses:
        Admission: 1. 78609
                    2. 7865
                    3. 78060
                    4. 08889
                    5. 2859
                    6. 7895
                    7. 486
                    8. 4280
                    9. 42731
                    10. 7807
        Release: all of the above and in addition 5990 & 514

        No strong correlation between the different features.
        Most hospitalizations were urgent and most were short, mainly a few days.
        No drastic amount of patients in any specific unit.
        Most hospitalization patients were excorted from home.
**Task 20: **
- Based on Doctor rank data(Senior/Not Senior), the ranks are split almost equally to 4  categories: Yes, No, ? and Depends from which date.
- Based on the PIE chart, They're split almost equally in amount of doctors.
- Based on Gradient Boosting Model that predicts rehospitalization based on Doctor rank, the accuracy is 50% meaniing we need Feature Extraction/Engineering in the preprocessing.
\end{verbatim}

מציאת קשר בין השכלה, מספר ילדים ומצב משפחתי לבין אישפוז חוזר בכל אחת
מהאינדיקציות 16-18 depends on: - {[}{[}Task 4{]}{]} - {[}{[}Task
16{]}{]}

\#Task

Data cleaning and completion for table: general data

\#Task

\begin{itemize}
\tightlist
\item
  {[}{[}Task 4{]}{]}
\item
  {[}{[}Task 28{]}{]}
\end{itemize}

\#Team

asdflkjqrgkajfdgqlekrjgqeropigjqegjqerpogijqe

\#Task

\subsubsection{Tasks:}\label{tasks-1}

\paragraph{Task 4: Data Cleaning}\label{task-4-data-cleaning}

\paragraph{Task 14: erBeforeHospitalization2
EDA}\label{task-14-erbeforehospitalization2-eda}

\paragraph{Task 14: erBeforeHospitalization2
EDA:}\label{task-14-erbeforehospitalization2-eda-1}

\paragraph{Task 16: optimal
distribution}\label{task-16-optimal-distribution}

\paragraph{Task 28: Find Connections}\label{task-28-find-connections}

\paragraph{Task 32: Time Series
Analysis}\label{task-32-time-series-analysis}

\subsubsection{Comments:}\label{comments}

For results and conclusions please download HTML file (not all plots can
be seen in ipynb file)

\#Team

\section{Team 09}\label{team-09}

We have implemented a CLI that enables to perform EDA on the file
\texttt{rehospitalization.xlsx}.\\
The file was originally uploaded to our course, in Moodle. The latest
version we have worked with is in the directory \texttt{./assets}.

Our EDA consists of a couple of missions: - \textbf{task\_06}: Missing
values treatment in the sheet \texttt{erBeforeHospitalization2}.\\
The output is available in the file
\texttt{team\_09\_task\_06\_erBeforeHospitalization.csv} (directory
\texttt{assets}). - \textbf{task\_15}: Parameter exploration in the
\texttt{hospitalization2.csv}. The results \& conclusions are available
in the document \texttt{team\_09\_task\_15\_hospitalization2\_EDA}
(directory \texttt{assets}). - \textbf{task\_22}: Relationship
exploration between the release day-of-week and rehospitalization. The
output is available in the file
\texttt{team\_09\_task\_22\_relationship\_day\_of\_week\_to\_rehospitalization.md}
(directory \texttt{assets}). - \textbf{task\_31}: Timeseries analysis
between 2nd admission date and rehospitalization occurrence. The output
is available in the file
\texttt{team\_09\_task\_31\_admission\_date\_timeseries\_analysis.md}
(directory \texttt{assets}).

\subsection{How to use the CLI?}\label{how-to-use-the-cli}

\subsubsection{Fill in missing values for
erBeforeHospitalization2}\label{fill-in-missing-values-for-erbeforehospitalization2}

The full path for \texttt{erBeforeHospitalization2} includes a number of
steps, each represented by a call to \texttt{main.py} below: - Transform
sheet \texttt{erBeforeHospitalization2} to \texttt{ASCII} encoding only
- Fill in missing values - Transform sheet
\texttt{erBeforeHospitalization2} back to the original encoding - Create
\texttt{erBeforeHospitalization2}

\begin{verbatim}
./main.py -v -i rehospitalization.xlsx -o rehospitalization.xlsx --ascii-encoded erBeforeHospitalization2 && \
./main.py -v -i rehospitalization.xlsx -o rehospitalization.xlsx --missing-values erBeforeHospitalization2 && \
./main.py -v -i rehospitalization.xlsx -o rehospitalization.xlsx --original-encoded erBeforeHospitalization2 && \
./main.py -v -i rehospitalization.xlsx -o team09_task06_erBeforeHospitalization.csv --sheet-file erBeforeHospitalization2
\end{verbatim}

\subsubsection{Relationship test between day of release and
rehospitalization}\label{relationship-test-between-day-of-release-and-rehospitalization}

\begin{verbatim}
./main.py -v -i rehospitalization.xlsx -o NA --relationship-test-release-date-rehospitalization
\end{verbatim}

\subsubsection{Time series analysis between 2nd admission date and
rehospitalization
occurrence}\label{time-series-analysis-between-2nd-admission-date-and-rehospitalization-occurrence}

\begin{verbatim}
./main.py -v -i rehospitalization.xlsx -o NA --time-series-analysis hospitalization2 Admission_Entry_Date
\end{verbatim}

\subsection{Note-worthy implementation
details}\label{note-worthy-implementation-details}

\subsubsection{task\_06: Missing values
treatment}\label{task_06-missing-values-treatment}

\texttt{erBeforeHospitalization2} sheet has many patients who were
admitted to the 2nd hospitalization without going through the
\texttt{ER} (\texttt{מיון}).\\
These patients lack details about \texttt{ER}, which led us to
supplement values that indicate that they did not visit the ER.\\
The parameters were chosen as following: - \texttt{Medical\_Record} =
\texttt{1000000} - \texttt{ev\_Admission\_Date} = \texttt{1900-01-01} -
\texttt{ev\_Release\_Time} = \texttt{1900-01-01} -
\texttt{Transport\_Means\_to\_ER} (\texttt{דרך\ הגעה\ למיון}) =
\texttt{\textquotesingle{}No\ Emergency\ Visit\textquotesingle{}} -
\texttt{ER} (\texttt{מיון}) =
\texttt{\textquotesingle{}No\ Emergency\ Visit\textquotesingle{}} -
\texttt{urgencyLevelTime} = \texttt{0} - \texttt{Diagnoses\_in\_ER}
(\texttt{אבחנות\ במיון}) = \texttt{0} - \texttt{codeDoctor} = \texttt{0}

Anyone who had a blank entry in the \texttt{Transport\_Means\_to\_ER}
(\texttt{דרך\ הגעה\ למיון}) column was updated with
\texttt{\textquotesingle{}Not\ provided\textquotesingle{}}.

For those in the \texttt{ER} (\texttt{מיון}) column with the value
\texttt{ICU} (\texttt{המחלקה\ לרפואה\ דחופה}) the missing values in the
columns \texttt{Diagnoses\_in\_ER} (\texttt{אבחנות\ במיון}) and
\texttt{codeDoctor} were updated with \texttt{1}.

\subsubsection{Non-ASCII chars}\label{non-ascii-chars}

Hebrew characters belong to a broader encoding family, \texttt{UTF-8}.\\
While it is widely used, ``best-practice'' recommends to avoid its usage
as it is impossible to know which 3rd party module will be used in the
system as a whole. On the contrary, \texttt{ASCII} encoding is supported
by virtually any 3rd party module. We have a dedicated mechanism, with
module \texttt{HebEngTranslator} at its core, that transforms
documents/tables to \texttt{ASCII} encoded only and back to the original
format.

\subsubsection{task\_15: Timeseries analysis between 2nd admission date
and rehospitalization
occurrence}\label{task_15-timeseries-analysis-between-2nd-admission-date-and-rehospitalization-occurrence}

\emph{Load \& Handle Missing Data}\\
Start by loading the dataset from a CSV file. Then count and print the
number of missing values (NaNs) per column. Handle missing data by
dropping rows with any missing values.

\emph{Define Column Types}\\
Dynamically identify numerical columns (\texttt{int64} and
\texttt{float64}) and categorical columns (\texttt{object} and
\texttt{category}).

\emph{Descriptive Statistics}\\
Print descriptive statistics for all columns, provide insights such as
\texttt{mean}, \texttt{median}, \texttt{count}. Provide
\texttt{standard\ deviation} insight for numerical data, count, unique
values, and top values for categorical data.

\emph{Visualization of Numerical Data}\\
* Histograms: Generate histograms for all numerical columns to visualize
their distributions. Use subplots to arrange these histograms in a grid
that adapts to the number of numerical columns.\\
* Box Plots: Create box plots for these columns to further analyze the
distribution of data and identify outliers. Visualization of Categorical
Data: * Bar Charts: For categorical data, generate bar charts to
visualize the frequency distribution of the top 10 most frequent
categories in each categorical column. Adjust the figure size, rotate
x-axis labels for readability, and ensure that the layout does not have
overlapping elements. Correlation Analysis: * Compute and visualize a
correlation matrix for numerical columns using a heatmap, which helps
identify any significant correlations between variables. Clustering of
Individual Variables: * Normalize the numerical data using
StandardScaler to prepare for clustering. Perform KMeans clustering for
each numerical column individually, visualizing the clustering result as
a scatter plot of the scaled data against its index, colored by cluster
label. This visualization is particularly useful for identifying
patterns or groups within individual features. Key Details: *
Matplotlib's subplots are used extensively to create grids of plots. The
grid size is dynamically adjusted based on the number of columns.
Scikit-learn's StandardScaler and KMeans are used for data scaling and
clustering, respectively. Seaborn is used for enhanced visualization
like box plots and heatmaps. DataFrame operations, such as selecting
data types, handling missing values, and indexing, are effectively
utilized to prepare and manipulate the data.

\subsubsection{task\_22: Relationship between release day of week and
re-hospitalization}\label{task_22-relationship-between-release-day-of-week-and-re-hospitalization}

None of the models should use this relationship.\\
There is a definitive bias towards finding predictive ability between
day of week and rehospitalization, as \emph{only patients who are
rehospitalized are mentioned in sheet \texttt{hospitalization1}}. We
have no data regarding patients who are not rehospitilized.\\
This prevents the \emph{mandatory establishment of relationship
existence} between day of week and rehospitalization, which makes any
predictive ability describe above invalid.

Here's the output of the relevant piece of logic to prove our statement:

\begin{verbatim}
(venv) maximc@Maxims-MacBook-Pro team_9 % ./main.py -i rehospitalization.xlsx -o NA --relationship-test-release-date-rehospitalization
Type of target variable: discrete.
        Possible target labels: "rehospitalized", "non-rehospitalized"
Type of feature variable: discrete.
        Possible target labels: "Sunday", "Monday", "Tuesday", "Wednesday", "Thursday", "Friday", "Saturday"
Conditions for statistical relationship test are not met, because of definitive bias:
        Number of rehospitilied patients: 7033 VS number of non-rehospitilized patients: 0
        We are unable to create "contingency table" that is a requirement for Chi-Squared or Fisher's Tests
(venv) maximc@Maxims-MacBook-Pro team_9 %
\end{verbatim}



\end{document}
